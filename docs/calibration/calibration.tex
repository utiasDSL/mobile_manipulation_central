\documentclass{article}

\usepackage[utf8]{inputenc}
\usepackage{parskip}
\usepackage{amssymb,amsfonts,amsmath,amscd}
\usepackage{bm}
\usepackage{hyperref}
\usepackage[pdftex]{graphicx}
\usepackage{url}
\usepackage[usenames,dvipsnames]{color}
\usepackage{enumitem}
\usepackage{mathtools}
\usepackage{float}

% for typesetting in-text numbers and units
\usepackage{siunitx}

% Table stuff.
\usepackage{booktabs}
\usepackage{makecell}
\usepackage{multirow}

% biblatex for bibliography
\usepackage[
backend=biber,
maxcitenames=1,
style=authoryear,
]{biblatex}

% \addbibresource{bibliography.bib}

\DeclareMathOperator*{\argmin}{argmin}

\title{Mobile Manipulation Calibration}
\author{Adam Heins}

\begin{document}

\maketitle

\section{Introduction}

The frames of interest are listed in Table~\ref{tab:frames}.

\begin{table}[h]
  \caption{Coordinate frames.}
  \centering
    \begin{tabular}{l c}
      \toprule
      Name & Subscript \\
      \midrule
      World        & $w$ \\
      Mobile base  & $b$ \\
      Base of arm  & $a$ \\
      End effector & $e$  \\
      Tool         & $t$ \\
      \bottomrule
    \end{tabular}
  \label{tab:frames}
\end{table}

The pose~$\bm{T}_{wt}$ of an arbitrary tool attached to the end effector can be
computed using the sequence of transforms
\begin{equation}\label{eq:kinematic_chain}
  \bm{T}_{wt} = \bm{T}_{wb}(\bm{q}_b)\bm{T}_{ba}\bm{T}_{ae}(\bm{q}_a)\bm{T}_{et},
\end{equation}
where~$\bm{T}_{wb}$ depends on the base
configuration~$\bm{q}_b=[x_b,y_b,\theta_b]^T$ and~$\bm{T}_{ae}$ depends on the
arm configuration~$\bm{q}_a$.

\section{Vicon Zero Pose}

The \texttt{vicon\_bridge} package allows one to provide a \emph{zero pose} to
specify the desired origin of a given model. Given the true robot frame~$\{r\}$
and the Vicon model frame~$\{v\}$, we have
\begin{equation*}
  \bm{T}_{wr} = \bm{T}_{wv}\bm{T}_{vr},
\end{equation*}
where~$\bm{T}_{vr}$ is a constant transform we need to calibrate. The zero
pose~$\bm{T}_{wv}^0$ is such that~$\bm{T}_{wr}=\bm{I}$, which
implies~$\bm{T}_{wv}^0\bm{T}_{vr}=\bm{I}$ and
thus
\begin{equation*}
  \bm{T}_{wv}^0 = \bm{T}_{vr}^{-1} = \begin{bmatrix} \bm{C}_{rv} & \bm{r}^{vr}_r \\ \bm{0}^T & 1 \end{bmatrix}
\end{equation*}

\section{Base Calibration}

\subsection{Center of Rotation}

We need to calibration the Vicon measured base poses~$\hat{\bm{T}}_{wv}$ so that
the origin of~$\bm{T}_{wb}$ is correct (i.e., we want it at the center of rotation, such that there is no translational motion when the base is commanded to rotate).
% We have
% \begin{equation}
%   \bm{T}_{wb}(\bm{q}_b) = \begin{bmatrix}
%     \bm{C}_{wb} & \bm{r}^{bw}_w \\
%     \bm{0}^T & 1
%   \end{bmatrix},
% \end{equation}
% where~$\bm{C}_{wb} = \bm{C}_z(\theta_b)\in SO(3)$ is a rotation about the $z$-axis
% and~$\bm{r}^{bw}_w=[x_b,y_b,z_b]^T$ with~$z_b$ a constant. Our goal in this
% section is to find the zero position~$\bm{r}^{vb}_b$ so that the point of
% rotation is correct.
% to calibrate the origin~$(x_b,y_b)$ so that it is located at the point of
% rotation of~$\theta_b$.
Starting at an arbitrary configuration~$\hat{\bm{T}}_{wv,0}$, we will move the base to
a sequence of desired yaw angles~$\theta^d_i$ and obtain the corresponding
measured configurations~$\hat{\bm{T}}_{wv,i}$. We want to find~$\bm{r}^{bv}_v$ such that~$\hat{\bm{r}}_{wb,i} = \hat{\bm{r}}_{wb,0}$
% \begin{equation}
%   \bm{r}^d_{b,i} = \hat{\bm{r}}_{b,i} + \Delta\bm{r}_b
% \end{equation}
is satisfied as closely as possible for each~$i$,
% where the left-hand side is
% \begin{equation*}
%   \bm{T}^d_{wb,i} = \begin{bmatrix} \bm{C}_z(\theta^d_i) & \bm{0} \\ \bm{0}^T & 1 \end{bmatrix}\hat{\bm{T}}_{wv,0} = \begin{bmatrix}
%   \bm{C}_z(\theta^d_i)\hat{\bm{C}}_{wv,0} & \bm{C}_z(\theta^d_i)\hat{\bm{r}}^{vw}_{w,0} \\ \bm{0}^T & 1
%   \end{bmatrix}
% \end{equation*}
% and the right-hand side is
% \begin{equation*}
% \hat{\bm{T}}_{wv,i}\bm{T}_{bv}^{-1} = \begin{bmatrix}
%     \hat{\bm{C}}_{wv,i} & \hat{\bm{C}}_{wv,i}\bm{r}^{bv}_v + \hat{\bm{r}}^{vw}_{w,i} \\ \bm{0}^{T} & 1
%   \end{bmatrix}.
% \end{equation*}
% We will assume
% that~$\hat{\bm{C}}_{wv,i}=\bm{C}_z(\theta^d_i)\hat{\bm{C}}_{wv,0}$ and look
% only at the translational part, which yields the least-squares problem
which yields the least-squares problem
\begin{equation}\label{eq:base_position_lstsq}
  \argmin_{\bm{r}^{bv}_v}\ \frac{1}{2}\sum_i\left\|\hat{\bm{C}}_{wv,i}\bm{r}^{bv}_v + \hat{\bm{r}}^{vw}_{w,i} - \hat{\bm{C}}_{wv,0}\hat{\bm{r}}^{bv}_{v} - \hat{\bm{r}}^{vw}_{w,0}\right\|^2.
\end{equation}
This gives us~$\bm{r}^{bv}_v$, but what we want for the Vicon zero pose is~$\bm{r}^{vb}_b=-\bm{C}_{bv}\bm{r}^{bv}_v$, so we need to negate the value and be careful to rotate it from the Vicon frame to the base frame if~$\bm{C}_{bv}\neq\bm{I}$.

We can write~\eqref{eq:base_position_lstsq} in the form~$\|\bm{A}\bm{x}-\bm{b}\|^2$ by taking~$\bm{x}=\bm{r}^{bv}_v$ and
\begin{align*}
  \bm{A} &= \begin{bmatrix}
    \hat{\bm{C}}_{wv,0} - \hat{\bm{C}}_{wv,0} \\ \vdots \\ \hat{\bm{C}}_{wv,n} - \hat{\bm{C}}_{wv,0}
  \end{bmatrix}, & \bm{b} = \begin{bmatrix}
  \hat{\bm{r}}^{vw}_{w,0} - \hat{\bm{r}}^{vw}_{w,0} \\ \vdots \\ \hat{\bm{r}}^{vw}_{w,0} - \hat{\bm{r}}^{vw}_{w,n}
  \end{bmatrix}.
\end{align*}

\subsection{Orientation (Yaw)}

\section{Arm--End Effector--Tool Calibration}

\begin{figure}[h]
  \centering
  \includegraphics[width=1\textwidth]{figures/robot_calibration.pdf}
  \caption{Frames on the robot being calibrated.}
    \label{fig:calibration}
\end{figure}

Having calibrated the base pose~$\bm{T}_{wb}$, we now wish to calibrate
(static) the transforms between the base and arm~$\bm{T}_{ba}$ and the EE and
tool~$\bm{T}_{et}$. We will assume that factory calibration for the arm
transform~$\bm{T}_{ae}(\bm{q}_a)$ is good enough.

To do so, we will collect a sequence of
pairs~$(\bm{q}_{a,i},\hat{\bm{T}}_{wt,i})$ by moving the arm to the sequence of
configurations~$\bm{q}_{a,i}$. We will then solve the (nonlinear) least squares
problem
\begin{equation}
  \argmin_{\bm{T}_{ba},\bm{T}_{wt}}\ \frac{1}{2}\sum_i\|\boxminus(\bm{T}_{wt}(\bm{q}),\hat{\bm{T}}_{wt})\|^2.
\end{equation}
over the manifold~$SE(3)$, where~$\bm{T}_{wt}$ is computed using~\eqref{eq:kinematic_chain} and
\begin{equation}
  \boxminus(\bm{T}_1,\bm{T}_2) = \log(\bm{T}_1^{-1}\bm{T}_2)^\vee
\end{equation}
is the error between the poses. We can include any prior information about the
transforms as an initial guess for the solver.


\end{document}
